\chapter{Social Engineering}
Social engineering adalah salah satu cara untuk mendapatkan data
(biasanya ilegal) tanpa menggunakan cara yang terlalu teknis.
Contoh data yang dimaksud adalah nama dan {\em password kita}

Social engineering umumnya memanfaatkan kelemahan manusia.
Sebagai contoh, kita diajak {\em ngobrol} dan tanpa kita sadari
kita menceritakan tentang nama dan password (atau nomor PIN) kita.
Beberapa contoh dari {\em social engineering} antara lain adalah
hal-hal tersebut di bawah ini.

\begin{itemize}
    \item Mengirimkan email palu {\em phising}, yang mengatakan bahwa
    sistem mengalami perubahan sehingga Anda diharapkan menggantikan
    password atau melakukan konfirmasi ulang mengenai passwordnya.
    Target kemudian diarahkan kepada sebuah situs web gadungan
    untuk memberikan nama (userid) dan password. Penjahat kemudian
    mendapatkan userid dan password dari target.
    \item Target ditelepon yang mengatasnamakan sebuah layanan
    (bank, layanan ecommerce, jasa angkutan online, dan sejenisnya)
    dan mengatakan bahwa target memenangkan sebuah undian.
    Untuk memastikan bahwa undian diklaim, target diminta untuk 
    menyebutkan PIN yang dikirimkan melalui SMS. Jika target menurut,
    maka penyerang akan masuk ke layanan yang bersangkutan dan 
    menggunakan data SMS itu untuk melakukan password reset
    (sesuai dengan password baru yang akan dia berikan).
    Penyerang kemudian akan mendapatkan akun kita dan menghabisi
    uang kita yang disimpan pada layanan tersebut.
\end{itemize}