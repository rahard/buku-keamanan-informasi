\chapter{Domain Name System (DNS)}
Salah satu komponen utama dari internet adalah penamaan domain, Domain Name
System (DNS). Penelusuran DNS juga merupakan salah satu langkah awal dari
evaluasi keamanan atau penyerangan. Mencari tahu sebuah target dimulai dari
layanan DNS dahulu. Sebagai contoh, jika akan dilakukan evaluasi keamanan dari
sebuah perusahaan maka yang pertama dilakukan adalah melakukan DNS query.

\section{Sejarah DNS}
Komputer bekerja dengan angka, atau lebih tepatnya dengan menggunakan bilangan
biner, ON dan OFF. Salah satu cara memberikan identitas komputer adalah dengan
menggunakan nomor IP. Contoh nomor IP antara lain dapat dilihat di bawah ini.
(Tahukah Anda itu nomor IP dari layanan apa?)

\begin{verbatim}
172.217.27.3
31.13.78.35
104.244.42.193
\end{verbatim}

Di sisi lain, manusia lebih mudah menghapal nama dibandingkan dengan angka.
Oleh sebab itu harus ada sebuah sistem yang melakukan konversi dari nama ke
angka dan sebaliknya. Pada awalnya, hal ini dapat dilakukan dengan menggunakan
sebuah tabel. Di sistem UNIX, tabel itu ada di berkas {\em /etc/hosts} yang
isinya serupa ini. Kolom kiri merupakan nomor IP (angka) dan kolom kanan berisi
nama untuk nomor IP tersebut. Nama boleh lebih dari satu.

\begin{verbatim}
167.205.24.34   paume www.paume.itb.ac.id
192.168.100.1   br accesspoint
\end{verbatim}

Pada awalnya, tabel ini jumlahnya tidak banyak sehingga dapat dikelola secara
manual. Inilah yang dilakukan oleh Jon Postel dan kawan-kawannya di Amerika
Serikat. Dikarenakan pengelolaannya menjadi semakin berat, maka dibentuklah
sebuah organisasi yang diberi nama {\em Internet Assigned Numbers Authority}
(IANA)\footnote{IANA juga yang mengatur pembagian atau distribusi nomor IP.}.
Tabel-tabel ini diperbaharui secara berkala dan dapat diunduh melalui FTP.

Jumlah komputer dan organisasi yang terhubung ke internet semakin bertambah
sehingga pengelolaan nama yang tersentralisai ini menjadi makin rumit. Ditambah
lagi ada perebutan nama yang populer (seperti server, mail, dan seterusnya).
Nama yang sama tidak diperbolehkan dalam sistem yang menggunakan tabel
tersebut. Akhirnya disepakati harus ada sebuah sistem yang terdistribusi
pengelolaannya. Maka lahirlah DNS.

Konsep DNS mirip dengan penamaan jalan di dunia nyata yang digunakan untuk
mengirim surat melalui pos. Sebagai contoh, pak pos di Amerika Serikat tidak
perlu tahu ada nama jalan Ganesha di Bandung, misalnya. Dia cukup mengetahui
kemana untuk mengirimkan email yang ditujukan ke Bandung. Mari kita ambil
contoh alamat surat menyurat saya.

\begin{verbatim}
Budi Rahardjo
Gedung PAU, lantai 4
Institut Teknologi Bandung
Jl. Ganesha 10
Bandung 40132
Indonesia
\end{verbatim}

Jika seseorang ingin mengirimkan surat dari Amerika ke saya dengan alamat
seperti di atas, maka alurnya adalah sebagai berikut.

Pengirim akan memasukkan surat ke pos terdekat. Pak pos akan membaca alamat
dari belakang (dari paling bawah). Surat itu ditujukan ke Indonesia, maka pak
pos akan meneruskan surat tersebut ke kantor pos Amerika yang mengumpulkan
surat-surat yang ditujukan ke Indonesia. (Ini ada dimana?) Dari sana
surat-surat (termasuk surat yang ditujukan kepada saya tersebut) ke Jakarta
karena semua surat yang menuju Indonesia masuk ke kantor pos Jakarta. Di kantor
pos Jakarta, surat-surat dipilah-pilah. Surat-surat yang menuju ke Bandung akan
dikirimkan ke kantor pos Bandung. Di kantor pos Bandung, surat saya terlihat
memiliki kode pos 40132 (Bandung Utara), maka surat itu akan diteruskan kepada
bagian yang mengirimkan ke Bandung Utara. Surat tersebut kemudian diteruskan ke
ITB yang beralamat di Jalan Ganesha nomor 10. Di ITB, surat tersebut diteruskan
kembali ke Gedung PAU. Di gedung PAU, surat tersebut diteruskan ke lantai 4 dan
akhirnya ke kantor saya. Demikianlah perjalanan surat dari luar negeri ke
kantor saya.

Perlu diperhatikan bahwa sistem seperti ini {\em scalable}. Maksudnya, pos di
berbagai negara tidak perlu memiliki daftar semua jalan di negara lain.
Penamaan jalan dapat berubah-ubah secara lokal, tetapi surat dari luar negeri
akan sampai seolah-olah perubahan lokal tersebut diketahui di luar negeri.

Sistem DNS diimplementasikan dalam program BIND.
Untuk setiap tingkatan (level) dari nama domain ada {\em name server} yang
bertanggung jawab terhadap domain-domain di level tersebut. Level teratas (top
level domain) ditandai dengan titik (dot) yang tidak nampak. Level teratas ini
awalnya dikelola oleh IANA dan kemudian diteruskan oleh ICANN (The Internet 
Corporation for Assigned Names and Numbers)\footnote{Penjelasan mengenai
perpindahan dari IANA ke ICANN akan dibahas secara terpisah.}.

Mari kita ambil sebuah contoh. Misalnya Anda ingin melihat halaman web yang
memiliki alamat (nama) ``www.paume.itb.ac.id''. Komputer (atau web browser)
tidak dapat mengenali nama tersebut. Komputer harus melakukan konversi ke
angka. Langkah-langkah yang dilakukan adalah sebagai berikut.

\begin{enumerate}
   \item Bertanya kepada root name server (ICANN yang mengelola top level
      domain dot), siapa pengelola (atau name server) dari ``ID''. ICANN
      memiliki beberapa root name servers; a.root-servers.net sampai
      m.root-servers.net.  Salah satu dari root name server
      tersebut akan menjawab bahwa domain ``.ID'' ada di server-server berikut;
      (biasanya ada lebih dari satu name server)
      a.dns.id, b.dns.id, c.dns.id, e.dns.id, dan sec3.apnic.net.
   \item Kepada salah satu dari name server (NS) tersebut ditanyakan, siapa (name
      server mana) yang mengelola ``AC.ID''. Jawabannya adalah b.dns.id,
      c.dns.id, d.dns.id, dan e.dns.id.
   \item Kepada salah satu dari NS tersebut ditanyakan, siapa (name
      server mana) yang mengelola ``ITB.AC.ID''. Jawabannya adalah
      ns3.itb.ac.id, sns-pb.isc.org, ns1.ai3.net, ns1.itb.ac.id, dan
      ns2.itb.ac.id.
   \item Kepada salah satu NS tersebut ditanyakan siapa name server domain
      ``PAUME.ITB.AC.ID''. Jawabannya adalah www.vlsi.itb.ac.id, ns3.itb.ac.id,
      ns2.itb.ac.id, alliance.globalnetlink.com, ic.vlsi.itb.ac.id,
      mx.insan.co.id.
   \item Kepada salah satu NS tersebut ditanyakan, berapa nomor IP dari
      ``WWW.PAUME.ITB.AC.ID''. Jawabannya adalah 167.205.24.34. Setelah
      mendapatkan nomor IP tersebut maka komputer dapat melakukan tugasnya.
      (Atau kalau menggunakan web browser, maka browser akan mengunjungi
      halaman web yang memiliki nomor IP tersebut.)
\end{enumerate}

Demikianlah proses yang terjadi di belakang layar. Tentu saja jika proses itu
dilakukan sering (berulang-ulang), maka akan terjadi pemborosan peggunaan
jaringan dan akan memakan waktu. Untuk itu dibuatlah proses {\em caching} yang
mengingat-ingat beberapa domain yang ditanyakan terakhir sehingga proses {\em
query} tidak harus mengulang dari awal. Demikian pula, daripada setiap komputer
melakukan proses ini sendiri-sendiri, maka di berbagai organisasi (dan layanan)
disediakan layanan DNS server. Biasanya Anda memasukkan DNS server ini di
konfigurasi jaringan di komputer Anda. DNS server inilah yang akan melakukan
proses {\em query} dan {\em caching}. 

Pada penjelasan di atas dikatakan kita dapat bertanya tentang name server.
Bagaimana cara melakukannya? Ada beberapa {\em tools} yang dapat digunakan
untuk melakukan hal tersebut. Untuk sistem yang berbasis UNIX (termasuk Linux,
berbagai variasi dari BSD, dan Mac OS X) ada {\em nslookup}, {\em dig} dan {\em host}.
Berikut ini contoh sebuah sesi untuk mencari name server dari domain itb.ac.id
dengan menggunakan program {\em host}.

\begin{verbatim}
$ host -t ns itb.ac.id
itb.ac.id name sever ns3.itb.ac.id.
itb.ac.id name sever sns-ob.isc.org.
itb.ac.id name sever ns1.ai3.net.
itb.ac.id name sever ns1.itb.ac.id.
itb.ac.id name sever ns2.itb.ac.id.
\end{verbatim}

Keamanan dari sistem DNS bergantung kepada beberapa hal, salah satunya adalah
ketergantungan kepada DNS server. Jika DNS server memberikan jawaban yang salah
terhadap {\em query}, maka kita bisa diteruskan ke alamat yang salah. (Di
alamat yang palsu tersebut bisa dibuatkan halaman yang mirip dengan web aslinya
sehingga kita tertipu. Ini adalah salah satu contoh serangan terhadap DNS.)


\section{DNS Filtering}
Salah satu cara melakukan penapisan {\em filtering} di internet adalah melalui
DNS. Untuk domain-domain tertentu (yang ingin difilter), DNS tidak memberikan
jawaban meskipun server yang dituju tersebut tetap ada. Inilah salah satu
filtering yang lazim dilakukan.
Tentu saja ``solusi'' terhadap filtering dengan cara tersebut adalah dengan 
menggunakan DNS server yang lain (atau milik sendiri).

Pemaksaan untuk melakukan filtering DNS dapat dilakukan melalui kebijakan,
bahwa semua harus menggunakan DNS server yang sudah ditentukan. Cara yang lebih
teknis lagi dapat dilakukan dengan memaksakan port 53 yang digunakan ke DNS ke
server tertentu sehingga pengguna tidak dapat menggunakan DNS server selain
yang sudah ditentukan. Hal tentu saja masih dapat dilewati dengan menggunakan
layanan DNScrypt.


\section{DNS spoofing}
Pencarian (query) DNS dilakukan dengan berbagai cara. Cara yang laing lazim
adalah dengan menggunakan paket UDP di port 53. Salah satu serangan yang dapat
dilakukan adalah memantau port 53 (UDP) ini dan menjawab dengan segera {\em
request} tersebut (dengan jawaban palsu). (Aspek kecepatan merupakan hal yang
utama untuk mencapai kesuksesan serangan ini.)

Program {\em dnsspoof}, yang merupakan bagian dari paket {\em dsniff}, dapat
digunakan untuk melakukan proses spoofing tersebut. Hal yang perlu diperhatikan
adalah bahwa lalu lintas (traffic) DNS harus terlihat oleh mesin yang melakukan
spoofing tersebut. Untuk menguji hal ini dapat digunakan {\em tcpdump} terlebih
dahulu. Jika memang traffic DNS dapat didengar, maka secara teori dia dapat
juga dipalsukan (spoofed). Jika traffic DNS tidak didengar, maka harus mencari
cara lain untuk melakukan DNS spoofing ini.

Misal kita memiliki berkas yang berisi daftar hosts yang ingin kita sesatkan
jawabannya. Berkas ini misalnya bernama {\em /tmp/hosts-spoofed} dan berisi
seperti contoh berikut.

\begin{verbatim}
$ cat /tmp/hosts-spoofed
192.168.1.101 ibanking.co.id
192.168.1.102 populer.co.id

$ dnsspoof -f /tmp/hosts-spoofed
\end{verbatim}

Coba melakukan query ke alamat ibanking.co.id. Jawabannya adalah 192.168.1.101.
Jika kita memasang server web di mesin tersebut dan membuat halaman palsu yang
mirip dengan halaman ibanking.co.id, maka kita dapat melakukan penipuan.

\section{Zone Transfer}
Daftar semua mesin (komputer) dan subdomain di dalam sebuah database DNS tidak
untuk diberikan ke semua orang. Biasanya {\em query} dibatasi untuk hal-hal
tertentu yang sangat spesifik, misalnya menanyakan nomor IP atau {\em name
server}. Jarang ada permintaan untuk melihat seluruh isi dari sebuah zona DNS.

Untuk melihat daftar seluruh domain tertentu dilakukan proses {\em zone
transfer}. Berikut ini contoh sebuah sesi {\em zone transfer} terhadap domain
itb.ac.id dengan menggunakan program {\em host} kepada NS ns3.itb.ac.id. Perlu
dicatat bahwa perintah berikut ini hanya dapat bekerja di lingkungan jaringan
ITB karena {\em zone transfer} dibatasi.

\begin{verbatim}
$ host -l itb.ac.id ns3.itb.ac.id
\end{verbatim}

Pembatasan {\em zone transfer} dilakukan karena dia dapat menghabiskan {\em
bandwidth} dan juga memberikan informasi yang berlebihan. Sebuah {\em query}
DNS dapat menghasilkan jawaban yang cukup panjang (besar, secara ukuran) bila
daftar domain yang ada cukup banyak. Ada faktor amplifikasi dari penggunaan
jaringan. Inilah yang ditargetkan untuk diserang. Di beberapa tempat,
kebocoran data melalui {\em zone transfer} dapat digunakan sebagai langkah 
awal untuk melakukan penyerangan terhadap sebuah instansi tertentu. Untuk itu,
zone transfer seharusnya dibatasi.

\begin{mdframed}[backgroundcolor=blue!20,linewidth=0pt,innerrightmargin=50pt]
   \begin{ExerciseList}
      \Exercise[title=Eksplorasi DNS.] 

      { } Pilih tiga buah domain (misal itb.ac.id,
      kemudian dua domain lain). Untuk setiap domain tersebut, lakukan
      hal-hal di bawah ini.
      \Question{Carilah server DNS (name server) dari domain tersebut.
      Ada berapa name server? Jika NS hanya ada satu, maka ini dianggap kurang baik.}
      \Question{Apalah Anda dapat melakukan zone transfer terhadap domain
      tersebut? Untuk sistem yang baik, biasanya Anda akan ditolak (REFUSED).
      Cari domain yang memperkenankan zone transfer.}
      \Question{Cari MX record dari domain tersebut.}
   \end{ExerciseList}
   
   \end{mdframed}

\section{Random Label Attack}
{\em Query} terhadap DNS dijawab oleh DNS {\em resolver}, yang biasanya kita
kenal dengan istilah ``DNS server'' dalam konfigurasi komputer kita. Domain
yang sering ditanyakan kemungkinan juga akan disimpan ({\em cached}) dalam
mesin resolver ini agar penggunaan jaringan dapat dikurangi. Pendekatan ini
juga menguntungkan name server karena ini akan mengurangi jumlah {\em query}.

{\em Random label attack} adalah penyerangan terhadap DNS server dengan cara
melakukan {\em query} kepada sebuah domain yang tidak ada.
Nama mesin dibuat secara random, misalnya ``xyz12345'',
dan kemudian dilakukan {\em query} terhadap nama mesin ini.
Nama yang random ini tentunya tidak akan ada dalam cache dan harus 
ditanyakan kepada name server. Hal ini
menyebabkan beban dari name server bertambah. Jika serangan ini dilakukan
secara bertubi-tubi dan dari tempat yang jumlahnya banyak, maka serangan ini
dapat menyebabkan {\em Denial of Service} (DOS) dari mesin name server.
