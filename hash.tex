\section{Fungsi Hash}
Pada bagian terdahulu sudah kita lihat bagaimana enkripsi (dan dekripsi) dapat digunakan untuk mengimplementasikan kerahasiaan ({\em confidentiality}). Pada bagian ini akan dijelaskan bagaimana kita dapat mengimplementasikan aspek {\em integrity}.

Ada situasi dimana kita ingin memastikan bahwa data kita tidak berubah (tanpa ijin yang sah). Datanya sendiri boleh terlihat (tidak rahasia) atau boleh juga dienkripsi. Yang utama adalah bahwa data tidak boleh berubah. Hal ini dapat dipenuhi dengan fungsi {\em hash}.

Sebagai contoh, kita ingin mengirimkan data berupa kata ``BUDI''. Kita ingin memastikan bahwa kata tersebut tidak berubah. Salah satu cara yang dapat kita lakukan adalah dengan ``menjumlahkan'' huruf-huruf dari kata tersebut. Misal, huruf A kita beri nilai 1, huruf B bernilai 2, dan seterusnya sampai huruf Z\footnote{Cara lain dari konversi dari huruf ke angka adalah dengan menggunakan nilai ASCII dari huruf tersebut. A=65, B=66, dan seterusnya. Cara ini dapat digunakan untuk melakukan hash terhadap berkas biner.}. Maka kata ``BUDI'' memiliki nilai 36.

\begin{verbatim}
B (2) + U (21)+ D (4) + I (9) => 36
\end{verbatim}

Jadi, data dapat kita kirimkan atau simpan dengan hasil ``hash'' tersebut, \{BUDI,36\}. (Biasanya pengiriman hasil hash dipisahkan untuk menghindari penyerangan. Hasil hash dapat juga dienkripsi untuk menjadi bagian dari tanda tangan digital, tetapi ini nanti kita bahas secara terpisah.)

Dalam contoh di atas, penjumlahan tersebut dapat membedakan antara ``BUDI'' dan ``RUDI'' karena ``RUDI'' akan memiliki nilai penjumlahan 52. Tetapi fungsi hash kita ini masih rentan terhadap serangan yang mengubah urutan. ``BUDI'' dan ``DUBI'' akan memiliki nilai hash yang sama, yaitu 36. Ini disebut {\em collision}. Maka harus dicari fungsi hash yang lebih bagus lagi.

Sebagai contoh, daripada hanya sekedar menjumlahkan huruf-huruf, kita juga dapat mempertimbangkan perbedaan antara huruf dan hasil sebelumnya. Misalnya rumus perjumlahan kita ganti menjadi seperti berikut:

\begin{verbatim}
jumlah = (karakter) OR (karakter sebelumnya) XOR (jumlah sebelumnya)
\end{verbatim}

Kita dapat membuat kode dalam bahasa Python seperti contoh berikut ini. 

\begin{python}
def hash2 (kata):
	# cara 2: mempertimbangkan karakter sebelumnya
	jumlah=0
	nilaic=0
	sebelumnya=0
	selisih=0
	for c in kata:
		nilaic = ord(c)
		selisih = nilaic | sebelumnya ^ jumlah
		jumlah += nilaic + selisih
		# buka komentar di bawah jika ingin melihat proses
		# print c, ord(c), jumlah, selisih
		sebelumnya = nilaic
	print "Hasil hash2 = ", jumlah
\end{python}

Algoritma hash yang baru di atas akan menghasilkan jumlah yang berbeda untuk BUDI (2047), IDUB (2091), dan DUBI (2077). Tentu saja algoritma ini masih rentan terhadap {\em collision} lainnya, tetapi ini hanya sebagai contoh bagaimana kita mengembangkan fungsi {\em hash} yang lebih baik dari sekedar penjumlahan huruf saja. ``Sejarah'' dari penjumlahan sebelumnya dapat menjadi bagian dari algoritmanya.

Fungsi hash sering disebut juga fungsi satu arah, karena jika kita diberikan hasil hash-nya maka kita tidak dapat mengetahui data awalnya. Dalam contoh kita, jika kita diberi angka ``36'' maka kita tidak dapat mengetahui kata awalnya. Ada terlalu banyak kemungkinan huruf-huruf yang dapat diatur sehingga memberikan nilai 36 itu. Sebagai contoh, ``DDDDDDDDD'' (huruf D sebanyak 9 kali) akan memberikan jumlah 36 juga.

Saat ini ada beberapa fungsi hash yang lazim digunakan, antara lain MD5, dan SHA. Sedikit tentang MD5, saat ini sudah ditemukan {\em collision} terhadap MD5 sehingga penggunaannya sudah dianggap tidak aman lagi. [Referensi]

Fungsi hash ini banyak digunakan untuk menjaga integritas dari data transaksi. Sebagai contoh jika kita dapat melindungi data transaksi (``KIRIM DARI AKUN A KE AKUN B DENGAN NILAI TRANSAKSI 1234567'') dari perubahan (akun yang diubah atau nilainya yang diubah). Fungsi hash juga merupakan salah satu kunci utam dalam algoritma Blockchain yang digunakan oleh Bitcoin atau {\em crypto currency} lainnya.

\section{MD5}
Salah satu fungsi hash yang masih lazim digunakan (meskipun bermasalah) adalah MD5. Fungsi MD5 ini merupakan karya dari Ronald Rivest di tahun 1991 sebagai pengganti MD4, yang juga merupakan karyanya.

MD5 dijabarkan dalam dokumen RFC~1321\cite{MD5} dan menghasilkan keluaran (nilai hash) sepanjang 128-bit. (Jika ditampilkan dalam bentuk heksadesimal, akan ada 32 karakter.) Perubahan sedikit saja akan mengubah nilai hash secara signifikan, sebagai mana dicontohkan seperti berikut.

\begin{mdframed}
\begin{verbatim}
$ md5
ini adalah percobaan
8459571a31ac4849ee7bf887f16f44c8
$ md5
ini adalah percobaan2
4b9547c47234271d8aa055c0572cb999
\end{verbatim}
\end{mdframed}

\begin{mdframed}
\begin{ExerciseList}
   \Exercise[title=MD5]
   \Question{Gunakan MD5 untuk menghasilkan hash dari sebuah berkas gambar (misal dalam format JPG). Ubah berkas gambar tersebut dengan sedikit modifikasi minor. Gunakan MD5 kembali untuk melihat nilai hash-nya.}
\end{ExerciseList}
\end{mdframed}

MD5 banyak digunakan karena tersedia untuk semua {\em platform} (sistem operasi). Saat ini MD5 sudah dianggap tidak aman karena sudah ada banyak bukti yang menunjukkan terjadinya {\em collision}. Kelemahan MD5 ini juga dimanfaatkan oleh malware Flame untuk membuat signature palsu.